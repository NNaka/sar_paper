% Document template based on LNCS, adapted by Matt Welsh <mdw@cs.berkeley.edu> and Mark Hempstead<mhempstead@coe.drexel.edu>

% This version is adapted to use PDFTEX to render to PDF directly. 
% If you want to use dvi, you need to change any figures to use '.eps'
% rather than '.pdf', and probably get rid of the hyperref package.

\documentclass{article}
\usepackage{program}
\usepackage{acm-style10} % ACM proceedings formatting
\usepackage{times}       % Use Adobe Times font set
\usepackage{epsfig,twocolumn}
\usepackage{url}
\usepackage[english]{babel} % mdw: Required to get good hyphenation on RH6.0
                            % (fixed in RH6.1)
\usepackage{graphicx} 
\usepackage{color}

\usepackage{caption}
\usepackage[labelformat=simple]{subcaption}
\renewcommand\thesubfigure{(\alph{subfigure})}
\graphicspath{{./figures/}}

% DO NOT EDIT THE BELOW BLOCK OF CODE
\def\dsp{\def\baselinestretch{1.10}}
\dsp{}
\newcommand{\XXXnote}[1]{{\bf\color{red} XXX:\@ #1}}
\setlength{\textheight}{9.25in}
\setlength{\columnsep}{0.33in}
\setlength{\textwidth}{7.4in}
\setlength{\footskip}{0.0in}
\setlength{\topmargin}{-0.25in}
\setlength{\headheight}{0.0in}
\setlength{\headsep}{0.0in}
\setlength{\oddsidemargin}{-.45in}
\setlength{\parindent}{1pc}
\pagestyle{empty}
\begin{document}
\date{}

% Title here
\title{\ttlfnt{GPU Accelerated Entropy Minimization for 3D SAR Images}}

% Author names, affiliations, and e-mail below
\author{Joshua Pfosi\\joshua.pfosi@tufts.edu
\and Norihito Naka\\norihito.naka@tufts.edu
\and Eric Miller\\eric.miller@tufts.edu}

\maketitle
%\copyrightspace
\thispagestyle{empty}

\subsection*{Abstract}
\begin{small}
  Synthetic aperture radar (SAR) imagery has many applications for
  reconnaissance and disaster relief.  As applications proliferate, the need for
  efficient processing of SAR data grows. A ubiquitous step in SAR image
  formation is autofocus. Much research has explored effective and efficient
  methods for autofocus with varying degrees of success. This paper builds upon
  these efforts by providing efficient implementations of one such algorithm
  using parallel architectures.  Both a parallel CPU as well as GPU
  implementation are compared to a standard MATLAB implementation yielding a
  significant improvement in performance.
\end{small}

\section{Introduction}\label{sec:introduction}

\begin{itemize}
  \item SAR images are used in applications
  \item Due to inherent errors, autofocus is often used to produce clearer images
  \item Efficient algorithms based on PGA have been developed (citation) but for some applications these methods are insufficient
  \item Ash (citation) details a gradient descent algorithm which, while much more computationally complex, can yield better results
  \item Ian and Colin generalized this technique to three dimensional SAR images but have major performance bottlenecks
  \item Their work used a brute force algorithm which, while effective, was very complex, slow, and implemented in MATLAB
\end{itemize}

\subsection{Contributions}

\begin{itemize}
  \item This work contributes two parallel algorithms for minimizing image entropy as a function of the image pulse history
  \item C++ multithreaded implementation (on CPU)
  \item CUDA implementation (on GPU)
\end{itemize}

\section{Related Work}\label{sec:relatedwork}

SAR image processing is a computationally burdensome task. Many researches have
developed techniques to optimize both image formation and autofocus techniques.
In particular, many papers have been written exploring optimizations to the
image formation technique~\cite{yegulalp1999fast, hartley, liu, clemente, fasih, jin,
park2013efficient}. Many use GPUs or multiple cores to achieve better
performance. Significantly less work has been done on autofocus techniques.
Passerone \textit{et al.} propose a CUDA based GPU implementation of two
autofocus algorithms, Range-Doppler and $\omega$-k with promising
results~\cite{gpu-sar}. Other work, such as~\cite{less_mem_high_eff_autofocus},
uses less of the image to obtain phase estimates. This reduces time and space
complexity but at the cost of efficacy. This work continues in the direction set
by~\cite{gpu-sar} by applying parallel architectures to a different autofocus
technique: gradient descent.

\section{Implementation Details}\label{sec:implementation}

Gradient descent entropy minimization is an instance of a ``matric-based''
algorithm. Metric-based autofocus algorithms accept a complex image represented
as an array of pulse contributions and obtain a phase offset vector which, when
applied to the original pulses, optimizes a given metric. We can express this
precisely by considering that in backprojection each radar pulse, $\vec{b_i}$,
contributes information to each pixel, $z_i$, to form a final image. To model
this, let each $\vec{b_i}$ be a matrix whose value at some point $(x,y)$
corresponds to the contribution of pulse $i$ to the pixel at $(x,y)$.  For $K$
pulses we have the pulse set $\{\vec{b}_1, \vec{b}_2, \dots \vec{b}_K\}$,
the sum of which forms the image matrix $\zb$.

Consider an image of $N$ pixels. Each pixel can be computed as:

\begin{equation}\label{eq:complex_image}
z_n = \sum_{i=1}^{K} b_{i,n}
\end{equation}

where $b_{i,n}$ is the $n$-th pixel for pulse $i$. As discussed, SAR data are
plagued by phase errors which can be modeled as per-pulse phase shifts:
$\phib = \{\phi_1, \phi_2, \dots \phi_k\}$. Applying this to
Eq.~\ref{eq:complex_image} yields:

\begin{equation}\label{eq:phase_complex_image}
z_n = \sum_{i=1}^{k} b_{i,n}e^{-j\phi_i}
\end{equation}

The goal of gradient descent autofocus is to optimize over a image quality
metric. We use image entropy, defined as~\cite{kragh2006monotonic}:

\begin{equation}\label{eq:entropy}
  H(\zb) = \sum_{n=1}^{N} \frac{|z_n|^2}{E_z} \ln
  \frac{|z_n|^2}{E_z}
  \text{,\indent} E(\zb) = \sum_{n=1}^{N} |z_n|^2.
\end{equation}

In Eq.~\ref{eq:entropy}, $\zb$ is a complex image, and $E_z$ is
the total image energy. We therefore seek:

\begin{equation}\label{eq:arg_min}
  \vec{\hat{\phi}} = \argmin(H(\zb(\phib)))
\end{equation}

Eq.~\ref{eq:arg_min} has no closed form solution~\cite{ash2012autofocus}. As
mentioned we employ gradient descent optimization. That is, we minimize the
objective function, $H$, first by forming an approximation to its gradient,
$\nabla H$. Next, we iterate down the negation of the gradient to find a
minimum. Therefore, the $l$-th iteration can be expressed as:

\begin{equation}\label{eq:recursion}
  \phib^l = \phib^{l-1} - s \nabla H(\zb(\phib^{l-1}))
\end{equation}

where $s$ is a scalar step size and the gradient of $H$ is with respect to
$\phib$. Eq.~\ref{eq:recursion} is evaluated by approximating the gradient
$\gb \equiv \nabla H(\phib^{l})$ via the following algorithm:

\begin{algorithm}
  \caption{Finite difference approximation of $\gb$}
  \label{alg:finitediff}
  Let $\eb_i$ be the $i$-th column of a $K \times K$ identity matrix.\\
  Let $\delta$ be some small offset.\\
  \ \\
  $H_0 \gets H(\zb(\phib^l))$\\
  \ \\
  \For{$i = 1,2,\dots, K$}{
    $H_i \gets H(\phib^l+\delta \times \eb_i)$\\
    $g_i \gets (H_i - H_0)/\delta$\\
  }
  \ \\
  \Return $\gb$
  \vspace{5 mm}
\end{algorithm}

The above computes the finite difference approximation to the gradient. By
iterating this recursive formulation until the image entropy converges, the
complex image can be computed via Eq.~\ref{eq:phase_complex_image} to yield a
focused image.

We exploit the data parallelism inherent in Alg.~\ref{alg:finitediff} in both
our reference MATLAB as well as C++ and CUDA implementations. Transitioning to a
C++ based solution offers much more efficient and finegrained thread control and
reduced memory footprint. Also, the performance difference between C++ and
MATLAB is well known. The next two subsections discuss the optimizations we
employed to improve performance and reduce complexity. 
 
\subsection{Reusing $\zb_{0}$}

The primary optimization we developed came from the ``reuse'' of the image
computed for $H_0$ in the first step of Alg.~\ref{alg:finitediff}. As shown by
Eq.~\ref{eq:phase_complex_image} and~\ref{eq:entropy}, computing $H_0$ requires
$\zb(\phib^l)$ as an intermediary value. In principle, this intermediary is
distinct for each $H_k$, but can be expressed by a function of $\phib^l$ and
$\delta$. That is, for each iteration of Alg.~\ref{alg:finitediff}, a given
$\phi_{i}$ is offset by $\delta$.  Substituting into
Eq.~\ref{eq:phase_complex_image} yields $z'_n$:

\begin{equation}\label{eq:z_prime}
  \begin{split}
    z'_n = z_n + b_{i,n}e^{-j(\phi_{i} + \delta)} - b_{i,n}e^{-j\phi_{i}} \\ =
    z_n + b_{i,n}e^{-j\phi_{i}}(e^{-j\delta} - 1)
  \end{split}
\end{equation}

Let $\alpha_{i} = e^{-j\phi_{i}}(e^{-j\delta} - 1)$. Thus, for each $H_i$, we
can express $\zb_{i} = \zb_{0} + \alpha_{i} \vec{b}_i$. Without this
optimization, Alg.~\ref{alg:finitediff} took $K$ iterations, each of which
summed over $N$ pixels, each requiring $K$ operations to evaluate, for a bound
of $\Theta(NK^2)$. Using this technique, a given pixel $z_i$ can be computed in
constant time after $H_0$ has been computed. This reduces to only $N$ operations
per iteration of Alg.~\ref{alg:finitediff} for $NK$ operations plus an
additional $NK$ operations to compute $H_0$, resulting in a complexity of only
$\Theta(NK)$.

\subsection{Reducing Memory Constraints}

Additionally, the expression for $H$ can be implemented more efficiently as:

\begin{equation}\label{eq:eff_entropy}
  H(\zb) = \frac{1}{E_z}(\sum_{n=1}^{N} |z_n|^2 \ln |z_n|^2 - E_z \ln E_z)
\end{equation}

In this form, Eq.~\ref{eq:eff_entropy} can be computed incrementally, building the
two summations concurrently as $\zb$ is evaluated. This allows the input phase
history to be partitioned at an arbitrary number of pixels, $N' \le N$, to match
the memory constraints of the system. We can, therefore, accumulate the summation
term and $E_z$ as each partition is processed. This was critical for the tight memory
limits of the GPU when image sizes exceeded the 2 GB of main memory. As
discussed in~\cite{gpu-sar} and shown by our results in
Section~\ref{sec:results}, the overhead of these data transfers can reduce
performance and so reducing the size of required temporary arrays alleviates
this issue.

\section{Results}\label{sec:Results}

\begin{itemize}
  \item Comparing performance between brute force, Matlab, C++, and CUDA implementations
    \subitem Graph of performance as a function of input size (see below)
    \subitem Graph of performance as a function of noise (to be generated)
    \subitem Graph of C++ performance as a function of threads (to be generated)
  \item Note the performance difference that almost exponentially grows with number of pulses
\end{itemize}

\section{Future Work}\label{sec:futurework}

\begin{itemize}
  \item Requires large amounts of RAM (10s to 100s of GB)  as algorithm must keep entire pulse history in memory
    \subitem Could map disk and focus the pulse history partially for memory constrained systems
  \item Memory requirements could also be reduced by pre-selecting a representative portion of the image and autofocusing only the subset
  \item Add support for multiple GPUs (fairly straightforward)
  \item Explore OpenMP and thread pooling to avoid overhead of thread spawning
\end{itemize}

\section{Conclusions}\label{sec:conclusions}

The optimized CUDA and C++ implementations for gradient descent are promising
in light of the need for more efficient methods of SAR image post-processing.
Both solutions we propose were found to be robust to varying degrees of noise,
and, barring memory constraints on the GPU, they scale efficiently with the
size of the pulse history through parallelizing computations and applying the
advantages of GPU and CPU architectures.


\begin{small}
\bibliographystyle{abbrv} \bibliography{sar_paper}
\end{small}

\end{document}

